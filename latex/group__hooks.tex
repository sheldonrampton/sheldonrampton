\hypertarget{group__hooks}{
\section{Hooks}
\label{group__hooks}\index{Hooks@{Hooks}}
}
\subsection*{Functions}
\begin{CompactItemize}
\item 
\hyperlink{group__hooks_g0d7a0b03039c561b63424b2a6cf6103f}{module\_\-hook} (\$module, \$hook)
\item 
\hyperlink{group__hooks_g55275aa404ced19ade543dec984cb93f}{module\_\-implements} (\$hook, \$sort=FALSE, \$refresh=FALSE)
\item 
\hyperlink{group__hooks_gbd6f189b7bcc74d05755d41ec1dfdfc3}{module\_\-invoke} ()
\item 
\hyperlink{group__hooks_g85e2028954b5e23c5ba2c5f1bd4e3e14}{module\_\-invoke\_\-all} ()
\item 
\hyperlink{group__hooks_gfea2ca26f6178230dca0a71ae21bef8b}{hook\_\-imagecache\_\-actions} ()
\item 
\hyperlink{group__hooks_gd1b2ea040501521cfab6683c33517998}{hook\_\-imagecache\_\-default\_\-presets} ()
\end{CompactItemize}


\subsection{Detailed Description}
Allow modules to interact with the Drupal core.

Drupal's module system is based on the concept of \char`\"{}hooks\char`\"{}. A hook is a PHP function that is named foo\_\-bar(), where \char`\"{}foo\char`\"{} is the name of the module (whose filename is thus foo.module) and \char`\"{}bar\char`\"{} is the name of the hook. Each hook has a defined set of parameters and a specified result type.

To extend Drupal, a module need simply implement a hook. When Drupal wishes to allow intervention from modules, it determines which modules implement a hook and calls that hook in all enabled modules that implement it.

The available hooks to implement are explained here in the Hooks section of the developer documentation. The string \char`\"{}hook\char`\"{} is used as a placeholder for the module name in the hook definitions. For example, if the module file is called example.module, then hook\_\-help() as implemented by that module would be defined as example\_\-help(). 

\subsection{Function Documentation}
\hypertarget{group__hooks_gfea2ca26f6178230dca0a71ae21bef8b}{
\index{hooks@{hooks}!hook\_\-imagecache\_\-actions@{hook\_\-imagecache\_\-actions}}
\index{hook\_\-imagecache\_\-actions@{hook\_\-imagecache\_\-actions}!hooks@{hooks}}
\subsubsection[{hook\_\-imagecache\_\-actions}]{\setlength{\rightskip}{0pt plus 5cm}hook\_\-imagecache\_\-actions ()}}
\label{group__hooks_gfea2ca26f6178230dca0a71ae21bef8b}


Inform ImageCache about actions that can be performed on an image.

\begin{Desc}
\item[Returns:]array An array of information on the actions implemented by a module. The array contains a sub-array for each action node type, with the machine-readable action name as the key. Each sub-array has up to 3 attributes. Possible attributes: \char`\"{}name\char`\"{}: the human-readable name of the action. Required. \char`\"{}description\char`\"{}: a brief description of the action. Required. \char`\"{}file\char`\"{}: the name of the include file the action can be found in relative to the implementing module's path. \end{Desc}
\hypertarget{group__hooks_gd1b2ea040501521cfab6683c33517998}{
\index{hooks@{hooks}!hook\_\-imagecache\_\-default\_\-presets@{hook\_\-imagecache\_\-default\_\-presets}}
\index{hook\_\-imagecache\_\-default\_\-presets@{hook\_\-imagecache\_\-default\_\-presets}!hooks@{hooks}}
\subsubsection[{hook\_\-imagecache\_\-default\_\-presets}]{\setlength{\rightskip}{0pt plus 5cm}hook\_\-imagecache\_\-default\_\-presets ()}}
\label{group__hooks_gd1b2ea040501521cfab6683c33517998}


Provides default ImageCache presets that can be overridden by site administrators.

\begin{Desc}
\item[Returns:]array An array of imagecache preset definitions. Each definition can be generated by exporting a preset from the database. Each preset definition should be keyed on its presetname (for easier interaction with drupal\_\-alter) and have the following attributes: \char`\"{}presetname\char`\"{}: the imagecache preset name. Required. \char`\"{}actions\char`\"{}: an array of action defintions for this preset. Required. \end{Desc}
\hypertarget{group__hooks_g0d7a0b03039c561b63424b2a6cf6103f}{
\index{hooks@{hooks}!module\_\-hook@{module\_\-hook}}
\index{module\_\-hook@{module\_\-hook}!hooks@{hooks}}
\subsubsection[{module\_\-hook}]{\setlength{\rightskip}{0pt plus 5cm}module\_\-hook (\$ {\em module}, \/  \$ {\em hook})}}
\label{group__hooks_g0d7a0b03039c561b63424b2a6cf6103f}


Determine whether a module implements a hook.

\begin{Desc}
\item[Parameters:]
\begin{description}
\item[{\em \$module}]The name of the module (without the .module extension). \item[{\em \$hook}]The name of the hook (e.g. \char`\"{}help\char`\"{} or \char`\"{}menu\char`\"{}). \end{description}
\end{Desc}
\begin{Desc}
\item[Returns:]TRUE if the module is both installed and enabled, and the hook is implemented in that module. \end{Desc}
\hypertarget{group__hooks_g55275aa404ced19ade543dec984cb93f}{
\index{hooks@{hooks}!module\_\-implements@{module\_\-implements}}
\index{module\_\-implements@{module\_\-implements}!hooks@{hooks}}
\subsubsection[{module\_\-implements}]{\setlength{\rightskip}{0pt plus 5cm}module\_\-implements (\$ {\em hook}, \/  \$ {\em sort} = {\tt FALSE}, \/  \$ {\em refresh} = {\tt FALSE})}}
\label{group__hooks_g55275aa404ced19ade543dec984cb93f}


Determine which modules are implementing a hook.

\begin{Desc}
\item[Parameters:]
\begin{description}
\item[{\em \$hook}]The name of the hook (e.g. \char`\"{}help\char`\"{} or \char`\"{}menu\char`\"{}). \item[{\em \$sort}]By default, modules are ordered by weight and filename, settings this option to TRUE, module list will be ordered by module name. \item[{\em \$refresh}]For internal use only: Whether to force the stored list of hook implementations to be regenerated (such as after enabling a new module, before processing hook\_\-enable). \end{description}
\end{Desc}
\begin{Desc}
\item[Returns:]An array with the names of the modules which are implementing this hook. \end{Desc}
\hypertarget{group__hooks_gbd6f189b7bcc74d05755d41ec1dfdfc3}{
\index{hooks@{hooks}!module\_\-invoke@{module\_\-invoke}}
\index{module\_\-invoke@{module\_\-invoke}!hooks@{hooks}}
\subsubsection[{module\_\-invoke}]{\setlength{\rightskip}{0pt plus 5cm}module\_\-invoke ()}}
\label{group__hooks_gbd6f189b7bcc74d05755d41ec1dfdfc3}


Invoke a hook in a particular module.

\begin{Desc}
\item[Parameters:]
\begin{description}
\item[{\em \$module}]The name of the module (without the .module extension). \item[{\em \$hook}]The name of the hook to invoke. \item[{\em ...}]Arguments to pass to the hook implementation. \end{description}
\end{Desc}
\begin{Desc}
\item[Returns:]The return value of the hook implementation. \end{Desc}
\hypertarget{group__hooks_g85e2028954b5e23c5ba2c5f1bd4e3e14}{
\index{hooks@{hooks}!module\_\-invoke\_\-all@{module\_\-invoke\_\-all}}
\index{module\_\-invoke\_\-all@{module\_\-invoke\_\-all}!hooks@{hooks}}
\subsubsection[{module\_\-invoke\_\-all}]{\setlength{\rightskip}{0pt plus 5cm}module\_\-invoke\_\-all ()}}
\label{group__hooks_g85e2028954b5e23c5ba2c5f1bd4e3e14}


Invoke a hook in all enabled modules that implement it.

\begin{Desc}
\item[Parameters:]
\begin{description}
\item[{\em \$hook}]The name of the hook to invoke. \item[{\em ...}]Arguments to pass to the hook. \end{description}
\end{Desc}
\begin{Desc}
\item[Returns:]An array of return values of the hook implementations. If modules return arrays from their implementations, those are merged into one array. \end{Desc}
