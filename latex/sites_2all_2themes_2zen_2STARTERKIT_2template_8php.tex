\hypertarget{sites_2all_2themes_2zen_2STARTERKIT_2template_8php}{
\section{sites/all/themes/zen/STARTERKIT/template.php File Reference}
\label{sites_2all_2themes_2zen_2STARTERKIT_2template_8php}\index{sites/all/themes/zen/STARTERKIT/template.php@{sites/all/themes/zen/STARTERKIT/template.php}}
}
\subsection*{Functions}
\begin{CompactItemize}
\item 
\hyperlink{sites_2all_2themes_2zen_2STARTERKIT_2template_8php_dd812d94a6ad6fa46e1ce89deac24684}{STARTERKIT\_\-theme} (\&\$existing, \$type, \$theme, \$path)
\end{CompactItemize}


\subsection{Detailed Description}
Contains theme override functions and preprocess functions for the theme.

ABOUT THE TEMPLATE.PHP FILE

The template.php file is one of the most useful files when creating or modifying Drupal themes. You can add new regions for block content, modify or override Drupal's theme functions, intercept or make additional variables available to your theme, and create custom PHP logic. For more information, please visit the Theme Developer's Guide on Drupal.org: \href{http://drupal.org/theme-guide}{\tt http://drupal.org/theme-guide}

OVERRIDING THEME FUNCTIONS

The Drupal theme system uses special theme functions to generate HTML output automatically. Often we wish to customize this HTML output. To do this, we have to override the theme function. You have to first find the theme function that generates the output, and then \char`\"{}catch\char`\"{} it and modify it here. The easiest way to do it is to copy the original function in its entirety and paste it here, changing the prefix from theme\_\- to STARTERKIT\_\-. For example:

original: \hyperlink{group__themeable_g499898a137ccb56620058a9ad884363a}{theme\_\-breadcrumb()} theme override: STARTERKIT\_\-breadcrumb()

where STARTERKIT is the name of your sub-theme. For example, the zen\_\-classic theme would define a zen\_\-classic\_\-breadcrumb() function.

If you would like to override any of the theme functions used in Zen core, you should first look at how Zen core implements those functions: theme\_\-breadcrumbs() in \hyperlink{sites_2all_2themes_2zen_2template_8php}{zen/template.php} \hyperlink{group__themeable_g77f59240b3aead5df8a777e4f69961b5}{theme\_\-menu\_\-item\_\-link()} in \hyperlink{sites_2all_2themes_2zen_2template_8php}{zen/template.php} \hyperlink{group__themeable_g11a9f127932a0b272cc0c0dabb4e7d0b}{theme\_\-menu\_\-local\_\-tasks()} in \hyperlink{sites_2all_2themes_2zen_2template_8php}{zen/template.php}

For more information, please visit the Theme Developer's Guide on Drupal.org: \href{http://drupal.org/node/173880}{\tt http://drupal.org/node/173880}

CREATE OR MODIFY VARIABLES FOR YOUR THEME

Each tpl.php template file has several variables which hold various pieces of content. You can modify those variables (or add new ones) before they are used in the template files by using preprocess functions.

This makes THEME\_\-preprocess\_\-HOOK() functions the most powerful functions available to themers.

It works by having one preprocess function for each template file or its derivatives (called template suggestions). For example: THEME\_\-preprocess\_\-page alters the variables for page.tpl.php THEME\_\-preprocess\_\-node alters the variables for node.tpl.php or for node-forum.tpl.php THEME\_\-preprocess\_\-comment alters the variables for comment.tpl.php THEME\_\-preprocess\_\-block alters the variables for block.tpl.php

For more information on preprocess functions and template suggestions, please visit the Theme Developer's Guide on Drupal.org: \href{http://drupal.org/node/223440}{\tt http://drupal.org/node/223440} and \href{http://drupal.org/node/190815#template-suggestions}{\tt http://drupal.org/node/190815\#template-suggestions} 

\subsection{Function Documentation}
\hypertarget{sites_2all_2themes_2zen_2STARTERKIT_2template_8php_dd812d94a6ad6fa46e1ce89deac24684}{
\index{sites/all/themes/zen/STARTERKIT/template.php@{sites/all/themes/zen/STARTERKIT/template.php}!STARTERKIT\_\-theme@{STARTERKIT\_\-theme}}
\index{STARTERKIT\_\-theme@{STARTERKIT\_\-theme}!sites/all/themes/zen/STARTERKIT/template.php@{sites/all/themes/zen/STARTERKIT/template.php}}
\subsubsection[{STARTERKIT\_\-theme}]{\setlength{\rightskip}{0pt plus 5cm}STARTERKIT\_\-theme (\&\$ {\em existing}, \/  \$ {\em type}, \/  \$ {\em theme}, \/  \$ {\em path})}}
\label{sites_2all_2themes_2zen_2STARTERKIT_2template_8php_dd812d94a6ad6fa46e1ce89deac24684}


Implementation of HOOK\_\-theme(). 