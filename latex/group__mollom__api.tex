\hypertarget{group__mollom__api}{
\section{Mollom API}
\label{group__mollom__api}\index{Mollom API@{Mollom API}}
}
\subsection*{Functions}
\begin{CompactItemize}
\item 
\hyperlink{group__mollom__api_gdd309b0309f638d79b17a686ea96f564}{hook\_\-mollom\_\-form\_\-list} ()
\item 
\hyperlink{group__mollom__api_ga5150fd4fabe048cbdeb41ec10908658}{hook\_\-mollom\_\-form\_\-info} (\$form\_\-id)
\item 
\hyperlink{group__mollom__api_gd0e440366ce3bd86317641b94693cbe6}{hook\_\-mollom\_\-form\_\-info\_\-alter} (\&\$form\_\-info, \$form\_\-id)
\end{CompactItemize}


\subsection{Detailed Description}
Functions to integrate with Mollom form protection.

In general, there are two different kinds of form submissions:\begin{itemize}
\item Entities created out of form submissions, which can be edited or deleted afterwards; whereas \char`\"{}entity\char`\"{} just refers to a uniquely identifiable data record.\item Form submissions that do not store any data, such as contact form mail messages and similar forms. While there may be an entity type (e.g., \char`\"{}contact\_\-mail\char`\"{}), there is no unique id for the post, which could be referred to later on.\end{itemize}


The Mollom API therefore supports two types of integration:\begin{itemize}
\item Entity form integration: Mollom integrates with the add/edit form for an entity, and additionally with the delete confirmation form of the entity to send feedback to Mollom. Almost everything happens in an automated way, solely based on the information provided via Mollom's info hooks, as explained below.\item Free integration: Mollom integrates with a given form\_\-id without 'entity'. Allowing users to send feedback requires to manually add \char`\"{}report to Mollom\char`\"{} links. Additionally requires to specify a 'report access \mbox{[}callback\mbox{]}' and 'report delete callback' to correctly handle access to report and delete a posted piece of content. An example for this kind of integration can be found in contact\_\-mollom\_\-form\_\-list(), mollom\_\-mail\_\-alter(), and related functions. This kind of integration is discouraged; it is recommended to implement and integrate with entity forms.\end{itemize}


Considering a very simple Instant Messaging module (\char`\"{}IM\char`\"{}) that implements a \char`\"{}im\_\-message\_\-form\char`\"{} allowing to send or edit an instant message, which should be possible to be protected by Mollom: 

\begin{Code}\begin{verbatim} function im_message_form(&$form_state, $im) {
   // To allow other modules to extend this form more easily and simplify our
   // own form submission handling, we use the dedicated parent key "im" for
   // all message properties (allows for easy casting from array to object).
   // Also helps us to explain handling of hierarchical sub-keys. :)
   $form['#tree'] = TRUE;

   // This is the stored message id (or 'post_id'), if any:
   // @see im_message_form_submit()
   $form['im']['id'] = array(
     '#type' => 'value',
     '#value' => isset($im->id) ? $im->id : NULL,
   );
   $form['im']['subject'] = array(
     '#type' => 'textfield',
     '#title' => t('Subject'),
     '#default_value' => isset($im->subject) ? $im->subject : '',
   );
   $form['im']['body'] = array(
     '#type' => 'textfield',
     '#title' => t('Message'),
     '#default_value' => isset($im->body) ? $im->body : '',
   );
   $form['actions']['submit'] = array(
     '#type' => 'submit',
     '#value' => t('Send'),
   );
   return $form;
 }
\end{verbatim}
\end{Code}



\char`\"{}entity\char`\"{} refers to an entity type. For example, \char`\"{}node\char`\"{}, \char`\"{}user\char`\"{}, \char`\"{}comment\char`\"{}, but also \char`\"{}webform\_\-submission\char`\"{}. It is not necessarily the name of a database table, but most often it actually is. The string is only used internally to identify to which module a form submission belongs. Once in use, it should not be changed.

Our form accepts an argument \$im, which we assume is the entity being created or edited, so we can also assume the following submit handler: 

\begin{Code}\begin{verbatim} function im_message_form_submit($form, &$form_state) {
   // Do whatever we need to do to insert or update the message.
   $im = (object) $form_state['values']['im'];
   im_save($im);
   // Ensure subsequent submit handlers have an entity id to work with, as
   // newly created messages will not have an id in the form values.
   $form_state['values']['im']['id'] = $im->id;
 }
\end{verbatim}
\end{Code}



The form values will not contain an entity id for a newly created message, which is usually an auto\_\-increment column value returned from the database. Whenever a form submission is related to the entity (e.g., leads to a stored entity being created, updated, or deleted) the form should $\ast$always$\ast$ contain the entity id in the same location of the submitted form values. Above example therefore purposively assigns the new id after inserting it.



\begin{Code}\begin{verbatim} function im_message_delete_confirm_form(&$form_state, $im) {
   $form['#im'] = $im;

   // Always provide entity id in the same form key as in the entity edit form.
   $form['im']['id'] = array('#type' => 'value', '#value' => $im->id);

   // In our case, we also need to enable #tree, so that above value ends up
   // in 'im][id' where we expect it.
   $form['#tree'] = TRUE;

   return confirm_form($form,
     t('Are you sure you want to delete %title?', array('%title' => $im->subject)),
     'im/' . $im->id,
     NULL,
     t('Delete')
   );
 }
\end{verbatim}
\end{Code}



The same applies to the delete confirmation form for the entity: it also provides the entity id for form submit handlers.

After ensuring these basics, the first step is to register the basic form\_\-id along with its title, entity type, as well as the form\_\-id of the corresponding delete confirmation form via \hyperlink{group__mollom__api_gdd309b0309f638d79b17a686ea96f564}{hook\_\-mollom\_\-form\_\-list()}:



\begin{Code}\begin{verbatim} function im_mollom_form_list() {
   $forms['im_message_form'] = array(
     'title' => t('Instant messaging form'),
     'entity' => 'im',
     // Specify the $form_id of the delete confirmation form that allows
     // privileged users to delete a stored message. Mollom will automatically
     // add form elements to send feedback to Mollom to this form.
     'delete form' => 'im_message_delete_confirm_form',
   );
   return $forms;
 }
\end{verbatim}
\end{Code}



Since modules can provide many forms, only minimal information is returned via \hyperlink{group__mollom__api_gdd309b0309f638d79b17a686ea96f564}{hook\_\-mollom\_\-form\_\-list()}. All details about the form are only required and asked for, if the site administrator actually enables Mollom's protection for the form. Therefore, everything else is registered via \hyperlink{group__mollom__api_ga5150fd4fabe048cbdeb41ec10908658}{hook\_\-mollom\_\-form\_\-info()}:



\begin{Code}\begin{verbatim} function im_mollom_form_info($form_id) {
   switch ($form_id) {
     case 'im_message_form':
       $form_info = array(
         // Optional: User permission list to skip Mollom's protection for.
         'bypass access' => array('administer instant messages'),
         // Optional: Function to invoke to put a bad form submission into a
         // moderation queue instead of discarding it.
         'moderation callback' => 'im_mollom_form_moderation',
         // Optional: To allow textual analysis of the form values, the form
         // elements needs to be registered individually. The keys are the
         // field keys in $form_state['values']. Sub-keys are noted using "]["
         // as delimiter.
         'elements' => array(
           'im][subject' => t('Subject'),
           'im][body' => t('Message body'),
         ),
         // Required when either specifying 'entity' or 'elements': the keys
         // are predefined data properties sent to Mollom (see full list in
         // hook_mollom_form_info()), the values refer to field keys in
         // $form_state['values']. Sub-keys are noted using "][" as delimiter.
         'mapping' => array(
           // Required when specifying 'entity' above: Where to find the id of
           // the entity being posted, edited, or deleted.
           // Important: The following assignment means that Mollom is able to
           // find the message id of the created, edited, or deleted message
           // in $form_state['values']['im']['id'].
           'post_id' => 'im][id',
           // Required if the form or entity contains a title-alike field:
           'post_title' => 'im][subject',
           // Optional: If our instant message form was accessible for
           // anonymous users and would contain form elements to enter the
           // sender's name, e-mail address, and web site, then those fields
           // should be additionally specified. Otherwise, information from
           // the global user session would be automatically taken over.
           'author_name' => 'im][sender][name',
           'author_mail' => 'im][sender][mail',
           'author_url' => 'im][sender][homepage',
         ),
       );
       break;
   }
   return $form_info;
 }
\end{verbatim}
\end{Code}



\char`\"{}elements\char`\"{} is a list of form elements, in which users can freely type text. The elements should not contain numeric or otherwise predefined option values, only text actually coming from user input. Only by registering \char`\"{}elements\char`\"{}, Mollom is able to perform textual analysis. Without registered form elements, Mollom can only provide a CAPTCHA.

\char`\"{}mapping\char`\"{} is a mapping of form elements to predefined XML-RPC data properties of the Mollom web service. For example, \char`\"{}post\_\-title\char`\"{}, \char`\"{}author\_\-name\char`\"{}, \char`\"{}author\_\-id\char`\"{}, \char`\"{}author\_\-mail\char`\"{}, etc. Normally, all form elements specified in \char`\"{}elements\char`\"{} would be merged into the \char`\"{}post\_\-body\char`\"{} data property. By specifying a \char`\"{}mapping\char`\"{}, certain form element values are sent for the specified data property instead. In our case, the form submission contains something along the lines of a title in the \char`\"{}subject\char`\"{} field, so we map the \char`\"{}post\_\-title\char`\"{} data property to the \char`\"{}subject\char`\"{} field.

Additionally, the \char`\"{}post\_\-id\char`\"{} data property always needs to be mapped to a form element that holds the entity id.

When registering a 'moderation callback', then the registered function needs to be available when the form is validated, and it is responsible for changing the submitted form values in a way that results in an unpublished post ending up in a moderation queue: 

\begin{Code}\begin{verbatim} function im_mollom_form_moderation(&$form, &$form_state) {
   $form_state['values']['status'] = 0;
 }
\end{verbatim}
\end{Code}



\begin{Desc}
\item[See also:]mollom\_\-node 

mollom\_\-comment 

mollom\_\-user 

mollom\_\-contact \end{Desc}


\subsection{Function Documentation}
\hypertarget{group__mollom__api_ga5150fd4fabe048cbdeb41ec10908658}{
\index{mollom\_\-api@{mollom\_\-api}!hook\_\-mollom\_\-form\_\-info@{hook\_\-mollom\_\-form\_\-info}}
\index{hook\_\-mollom\_\-form\_\-info@{hook\_\-mollom\_\-form\_\-info}!mollom_api@{mollom\_\-api}}
\subsubsection[{hook\_\-mollom\_\-form\_\-info}]{\setlength{\rightskip}{0pt plus 5cm}hook\_\-mollom\_\-form\_\-info (\$ {\em form\_\-id})}}
\label{group__mollom__api_ga5150fd4fabe048cbdeb41ec10908658}


Return information about a form that can be protected by Mollom.

\begin{Desc}
\item[Parameters:]
\begin{description}
\item[{\em \$form\_\-id}]The form id to return information for.\end{description}
\end{Desc}
\begin{Desc}
\item[Returns:]An associative array describing the form identified by \$form\_\-id:\begin{itemize}
\item mode: (optional) The default protection mode for the form, which can be one of:\begin{itemize}
\item MOLLOM\_\-MODE\_\-ANALYSIS: Text analysis of submitted form values with fallback to CAPTCHA.\item MOLLOM\_\-MODE\_\-CAPTCHA: CAPTCHA-only protection.\end{itemize}
\item bypass access: (optional) A list of user permissions to check for the current user to determine whether to protect the form with Mollom or do not validate submitted form values. If the current user has at least one of the listed permissions, the form will not be protected.\item moderation callback: (optional) A function name to invoke when a form submission would normally be discarded. This allows modules to put such posts into a moderation queue (i.e., to accept but not publish them) by altering the \$form or \$form\_\-state that are passed by reference.\item mail ids: (optional) An array of mail IDs that will be sent as a result of this form being submitted. When these mails are sent, a 'report to Mollom' link will be included at the bottom of the mail body. Be sure to include only user-submitted mails and not any mails sent by Drupal since they should never be reported as spam.\item elements: (optional) An associative array of elements in the form that can be configured for Mollom's text analysis. The site administrator can only select the form elements to process (and exclude certain elements) when a form registers elements. Each key is a form API element parents string representation of the location of an element in the form. For example, a key of \char`\"{}myelement\char`\"{} denotes a form element value on the top-level of submitted form values. For nested elements, a key of \char`\"{}parent\mbox{]}\mbox{[}child\char`\"{} denotes that the value of 'child' is found below 'parent' in the submitted form values. Each value contains the form element label. If omitted, Mollom can only provide a CAPTCHA protection for the form.\item mapping: (optional) An associative array to explicitly map form elements (that have been specified in 'elements') to the data structure that is sent to Mollom for validation. The submitted form values of all mapped elements are not used for the post's body, so Mollom can validate certain values individually (such as the author's e-mail address). None of the mappings are required, but most implementations most likely want to at least denote the form element that contains the title of a post. The following mappings are possible:\begin{itemize}
\item post\_\-id: The form element value that denotes the ID of the content stored in the database.\item post\_\-title: The form element value that should be used as title.\item post\_\-body: Mollom automatically assigns this property based on all elements that have been selected for textual analysis in Mollom's administrative form configuration.\item author\_\-name: The form element value that should be used as author name.\item author\_\-mail: The form element value that should be used as the author's e-mail address.\item author\_\-url: The form element value that should be used as the author's homepage.\item author\_\-id: The form element value that should be used as the author's user uid.\item author\_\-openid: Mollom automatically assigns this property based on 'author\_\-id', if no explicit form element value mapping was specified.\item author\_\-ip: Mollom automatically assigns the user's IP address if no explicit form element value mapping was specified. \end{itemize}
\end{itemize}
\end{Desc}
\hypertarget{group__mollom__api_gd0e440366ce3bd86317641b94693cbe6}{
\index{mollom\_\-api@{mollom\_\-api}!hook\_\-mollom\_\-form\_\-info\_\-alter@{hook\_\-mollom\_\-form\_\-info\_\-alter}}
\index{hook\_\-mollom\_\-form\_\-info\_\-alter@{hook\_\-mollom\_\-form\_\-info\_\-alter}!mollom_api@{mollom\_\-api}}
\subsubsection[{hook\_\-mollom\_\-form\_\-info\_\-alter}]{\setlength{\rightskip}{0pt plus 5cm}hook\_\-mollom\_\-form\_\-info\_\-alter (\&\$ {\em form\_\-info}, \/  \$ {\em form\_\-id})}}
\label{group__mollom__api_gd0e440366ce3bd86317641b94693cbe6}


Alter registered information about a form that can be protected by Mollom.

\begin{Desc}
\item[Parameters:]
\begin{description}
\item[{\em \&\$form\_\-info}]An associative array describing the protectable form. See \hyperlink{group__mollom__api_ga5150fd4fabe048cbdeb41ec10908658}{hook\_\-mollom\_\-form\_\-info()} for details. \item[{\em \$form\_\-id}]The \$form\_\-id of the form. \end{description}
\end{Desc}
\hypertarget{group__mollom__api_gdd309b0309f638d79b17a686ea96f564}{
\index{mollom\_\-api@{mollom\_\-api}!hook\_\-mollom\_\-form\_\-list@{hook\_\-mollom\_\-form\_\-list}}
\index{hook\_\-mollom\_\-form\_\-list@{hook\_\-mollom\_\-form\_\-list}!mollom_api@{mollom\_\-api}}
\subsubsection[{hook\_\-mollom\_\-form\_\-list}]{\setlength{\rightskip}{0pt plus 5cm}hook\_\-mollom\_\-form\_\-list ()}}
\label{group__mollom__api_gdd309b0309f638d79b17a686ea96f564}


Return information about forms that can be protected by Mollom.

Mollom invokes this hook for all modules to gather information about forms that can be protected. Only forms that have been registered via this hook are configurable in Mollom's administration interface.

\begin{Desc}
\item[Returns:]An associative array containing information about the forms that can be protected, keyed by \$form\_\-id:\begin{itemize}
\item title: The human-readable name of the form.\item entity: (optional) The internal name of the entity type the form is for, e.g. 'node' or 'comment'. This is required for all forms that will store the submitted content persistently. It is only optional for forms that do not permanently store the submitted form values, such as contact forms that only send an e-mail, but do not store it in the database. Note that forms that specify 'entity' also need to specify 'post\_\-id' in the 'mapping' (see below).\item report access callback: (optional) A function name to invoke to check access to Mollom's dedicated \char`\"{}report to Mollom\char`\"{} form, which should return either TRUE or FALSE (like any other menu \char`\"{}access callback\char`\"{}).\item report access: (optional) A list containing user permission strings, from which the current user needs to have at least one. Should only be used if no \char`\"{}report access callback\char`\"{} was defined.\item report delete callback: (optional) A function name to invoke to delete an entity after reporting it to Mollom.\end{itemize}
\end{Desc}
\begin{Desc}
\item[See also:]\hyperlink{group__mollom__api_ga5150fd4fabe048cbdeb41ec10908658}{hook\_\-mollom\_\-form\_\-info()} \end{Desc}
