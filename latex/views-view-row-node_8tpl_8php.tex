\hypertarget{views-view-row-node_8tpl_8php}{
\section{sites/all/modules/contrib/views/theme/views-view-row-node.tpl.php File Reference}
\label{views-view-row-node_8tpl_8php}\index{sites/all/modules/contrib/views/theme/views-view-row-node.tpl.php@{sites/all/modules/contrib/views/theme/views-view-row-node.tpl.php}}
}
\subsection*{Variables}
\begin{CompactItemize}
\item 
\hypertarget{views-view-row-node_8tpl_8php_2b55f72c6da822668fde924c6af14b49}{
print \textbf{\$node}}
\label{views-view-row-node_8tpl_8php_2b55f72c6da822668fde924c6af14b49}

\item 
\hypertarget{views-view-row-node_8tpl_8php_e02fef85483c0eb837f4d4d00b9f0543}{
if(\$comments) \textbf{endif}}
\label{views-view-row-node_8tpl_8php_e02fef85483c0eb837f4d4d00b9f0543}

\end{CompactItemize}


\subsection{Detailed Description}
Default simple \hyperlink{classview}{view} template to display a single node.

Rather than doing anything with this particular template, it is more efficient to use a variant of the node.tpl.php based upon the \hyperlink{classview}{view}, which will be named node-view-VIEWNAME.tpl.php. This isn't actually a views template, which is why it's not used here, but is a template 'suggestion' given to the node template, and is used exactly the same as any other variant of the node template file, such as node-NODETYPE.tpl.php 